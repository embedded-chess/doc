% !TeX root = ../main.tex

\section{Einführung}

\begin{frame}{Projektidee}
    \begin{itemize}
        \item Simulation einer Schachfigur durch Dezibot
        \item Auswahl aus allen gängigen Schachfiguren
        \item Angabe einer Zug"=Reihenfolge
        \item Validierung des Zuges, Anzeige über LEDs
        \item Ausführung durch Dezibot auf Schachbrett
    \end{itemize}
\end{frame}


\begin{frame}{Projekt"=Rahmen}
    \begin{itemize}
        \item[$+$] Schwerpunkt auf \emph{einem} Dezibot\\
            $\Rightarrow$ jeweils eine Figur auf dem Brett\\
        
        \item[$+$] Implementierung der Grundlagen\\
        \begin{itemize}
            \item Zug"=Validierung aller Schachfigurtypen
            \item Bewegung auf Schachbrett
        \end{itemize}
    \end{itemize}

    \pause
    \begin{itemize}
        \item[$-$] Schlagen, Bewegung von Brett, Ausweichen anderer Dezibots
        \item[$-$] vollständige Schachbrett"=Logik
        \item[$-$] Interaktion mit anderen Dezibots

        \item[$-$] kontaktlose Verbindung zu Spielenden\\
            $\Rightarrow$ vordefinierte Zug"=Reihenfolge

        \item[$\Rightarrow$] Anschluss an andere Projekte möglich
    \end{itemize}
\end{frame}


\begin{frame}{Verweis auf Dokumentation}
    Weitere Inhalte der Projekt"=Dokumentation:

    \begin{itemize}
        \item Verwendung
        \item Projektstruktur
        \item Schachstrukturen \& -logik
        \item Bewegungsiterationen und weitere Ansätze
        \item Ausblick
    \end{itemize}
\end{frame}
