% !TeX root = ../main.tex

\section{Projektstruktur}
\label{sec:final-project-structure}

Unser Beitrag in diesem Projekt besteht einerseits in der Änderung und Ergänzung der bereits vorhandenen \texttt{dezibot}-Library~\cite{dezibotteamDezibotDezibot2025}, sowie dem Hinzufügen unseres \texttt{Embedded\-Chess\-Pieces}-Packages. Letzteres befindet sich nach Vorgabe der Dozenten unter \texttt{example/""Embedded\-Chess\-Pieces} (vgl. \autoref{fig:project-structure}).

\begin{figure}[h]
\centering
\begin{cminted}{text}
+-- src/              # dezibot/
|   +-- colorSensor/
|   +-- ...
|   `-- multiColorLight/
`-- example/EmbeddedChessPieces/
    +-- src
    |   +-- ECPChessLogic/
    |   +-- ECPColorDetection/
    |   +-- ECPMovement/
    |   `-- ECPSignalDetection/
    `-- examples/
        +-- playground/
        +-- showcase/
        `-- tests/
\end{cminted}
\caption{Vereinfachte Projekt"=Struktur}
\label{fig:project-structure}
\end{figure}

Die Business"=Logic ist im \texttt{src}-Verzeichnis des Projektes \texttt{example/""Embedded\-Chess\-Pieces} abgelegt. Darin sind die Strukturen und die Logik (siehe \autoref{sec:chess-logic}), welche dieses Projekt ermöglichen, abgelegt.

Unter \texttt{examples} befinden sich verschiedene Arduino-Sketches, u.a. zum Testen einzelner Funktionen. Außerdem sind dort die zwei wichtigen Sketches, welche für die finale Vorführung (\texttt{showcase}\footnote{\url{https://github.com/nicosrm/24-emb-chess/tree/main/example/EmbeddedChessPieces/examples/showcase}}) notwendig sind, abgelegt. Dabei handelt es sich um einen Infrarot-Emitter, welcher in \autoref{sec:movement-ir} erläutert wird, sowie um einen Sketch, der auf dem Dezibot läuft, welcher schlussendlich die Schachfigur simuliert. Zum Ausprobieren wurde weiterhin ein \texttt{playground}"=Sketch\footnote{\url{https://github.com/nicosrm/24-emb-chess/blob/main/example/EmbeddedChessPieces/examples/playground/playground.ino}} als Schritt"=für"=Schritt"=Anleitung hinzugefügt, mit welchem verschiedene implementierte Funktionen ausprobiert werden können.

Für die Einrichtung muss müssen alle notwendigen Third"=Party"=Packages installiert werden, welche für das \texttt{dezibot}"=Package notwendig sind. Zusätzlich muss ein symbolischer Link vom Projekt"=Repository sowie \texttt{example/""Embedded\-Chess\-Pieces} in das Arduino"=Library"=Verzeichnis gelegt werden. Weitere Informationen können aus der README\footnote{\url{https://github.com/nicosrm/24-emb-chess/blob/feature/37-final-example-and-readme/README.md\#installation}} entnommen werden.
