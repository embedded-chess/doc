% !TeX root = ../main.tex

\section{Projektstruktur}
\label{sec:final-project-structure}

Unser Beitrag in diesem Projekt besteht einerseits in der Änderung und Ergänzung der bereits vorhandenen \texttt{dezibot}-Library~\cite{dezibotteamDezibotDezibot2025}, sowie dem Hinzufügen unseres \texttt{EmbeddedChessPieces}-Packages. Letzteres befindet sich nach Vorgabe der Dozenten unter \texttt{example/EmbeddedChessPieces}.

Die Business"=Logic ist im \texttt{src}-Verzeichnis des Projektes abgelegt. Darin sind die Strukturen und die Logik (siehe \autoref{sec:chess-logic}), welche dieses Projekt ermöglichen, abgelegt.

Unter \texttt{examples} befinden sich verschiedene Arduino-Sketches, u.a. zum Testen einzelner Funktionen. Außerdem sind dort die zwei wichtigen Sketches, welche für die finale Vorführung (\texttt{showcase}\footnote{\url{https://github.com/nicosrm/24-emb-chess/tree/main/example/EmbeddedChessPieces/examples/showcase}}) notwendig sind, abgelegt. Dabei handelt es sich um einen Infrarot-Emitter, welcher in \autoref{sec:movement-ir} erläutert wird, sowie um einen Sketch, der auf dem Dezibot läuft, welcher schlussendlich die Schachfigur simuliert. Zum Ausprobieren wurde weiterhin ein \texttt{playground}"=Sketch\footnote{\url{https://github.com/nicosrm/24-emb-chess/blob/main/example/EmbeddedChessPieces/examples/playground/playground.ino}} als Schritt"=für"=Schritt"=Anleitung hinzugefügt, mit welchem verschiedene implementierte Funktionen ausprobiert werden können.

Für die Einrichtung muss müssen alle notwendigen Third-Party-Packages manuell installiert werden, welche für das \texttt{dezibot}-Package notwendig sind. Diese sind in der README unseres Projektes aufgelistet. Die Einrichtung der Arduino-IDE ändert sich durch dieses Projekt nicht. Zusätzlich muss ein symbolischer Link vom Projekt-Repository sowie \texttt{example/EmbeddedChessPieces} in das Arduino-Library-Verzeichnis gelegt werden.
