% !TeX root = ../main.tex

\section{Messungen Winkel}
\label{sec:measurements-dezibot-angle}

Um die in \autoref{tab:measurements-dezibot-angle} aufgelisteten Werte so genau wie möglich zu messen, wurde ein eigenes 3D\hyphen gedrucktes Modell entworfen. Dies vereinfacht das Ablesen der Winkel und verstärkt die Präzision.

\begin{table}[h!]
    \centering
    \begin{tabular}{r|rr}
    $\varphi_{B}$ & $\varphi_{D}$ & $\Delta\varphi$ \\ \hline
    %
    0°   & 0°   & 0°  \\
    30°  & 31°  & 1°  \\
    60°  & 60°  & 0°  \\
    90°  & 96°  & 6°  \\
    120° & 128° & 8°  \\
    150° & 148° & -2° \\
    180° & 182° & 2°  \\
    210° & 213° & 3°  \\
    240° & 243° & 3°  \\
    270° & 276° & 6°  \\
    300° & 305° & 5°  \\
    330° & 328° & -2° \\
    0°   & 0°   & 0°
\end{tabular}

    \caption{Signal-Messungen von \texttt{ECP\-Signal\-Detection::measure\-Dezibot\-Angle} (vgl. \autoref{sec:angle-determination}). $\varphi_B$ bzw. $\varphi_B$ sind Winkel, in denen Beacon bzw. Dezibot stehen. $\Delta\varphi=\varphi_D - \varphi_B$.}
    \label{tab:measurements-dezibot-angle}
\end{table}


\section{Messungen Infrarot-Rotation}
\label{sec:measurements-ir-rotation}

\begin{table}[h!]
    \centering
    \begin{tabular}{r||rrr||rrr}
    & \multicolumn{3}{c||}{Links} & \multicolumn{3}{c}{Rechts} \\ \hline
    $\varphi_{\text{initial}}$ & $\varphi_{\text{goal}}$ & $\varphi_{\text{end}}$ & $\Delta\varphi$ & $\varphi_{\text{goal}}$ & $\varphi_{\text{end}}$ & $\Delta\varphi$\\ \hline
    %
    0°   & 270°   & 260°    & 10°   & 90°   & 80°  & 10°  \\
    0°   & 270°   & 260°    & 10°   & 90°   & 75°  & 15°  \\
    0°   & 270°   & 262°    & 8°    & 90°   & 75°  & 15°  \\
    0°   & 270°   & 262°    & 8°    & 90°   & 72°  & 18°  \\
    0°   & 270°   & 260°    & 10°   & 90°   & 72°  & 18°  \\ \hline
    %
    90°  & 0°     & 5°      &-5°    & 180°  & 182° & -2°  \\
    90°  & 0°     & 10°     &-10°   & 180°  & 178° & 2°   \\
    90°  & 0°     & 12°     &-12°   & 180°  & 175° & 5°   \\
    90°  & 0°     & 10°     &-10°   & 180°  & 175° & 5°   \\
    90°  & 0°     & 8°      &-8°    & 180°  & 178° & 2°   \\ \hline
    %
    180° & 90°    & 88°    & 2°     & 270°  & 260°  & 10° \\
    180° & 90°    & 89°    & 1°     & 270°  & 260°  & 10° \\
    180° & 90°    & 92°    & -2°    & 270°  & 265°  & 5°  \\
    180° & 90°    & 88°    & 2°     & 270°  & 265°  & 5°  \\
    180° & 90°    & 88°    & 2°     & 270°  & 262°  & 8°  \\ \hline
    %
    270° & 180°   & 182°   & -2°    & 0°    & 10°   & -10° \\
    270° & 180°   & 180°   & 0°     & 0°    & 10°   & -10° \\
    270° & 180°   & 178°   & 2°     & 0°    & 10°   & -10° \\
    270° & 180°   & 182°   & -2°    & 0°    & 10°   & -10° \\
    270° & 180°   & 178°   & 2°     & 0°    & 12°   & -12°
\end{tabular}

    \caption{Messungen der Infrarot-basierten Rotation (vgl. \autoref{sec:movement-ir}). $\varphi_{\text{initial}}$: initialer Winkel, in dem Dezibot ausgerichtet ist. $\varphi_{\text{goal}}$: Ziel-Winkel, in dem Dezibot nach erfolgreicher Rotation ausgerichtet sein sollte. $\varphi_{\text{end}}$: Winkel, in dem Dezibot nach Rotation tatsächlich ausgerichtet war. $\Delta\varphi = \vert\varphi_{\text{end}}-\varphi_{\text{goal}}\vert$.}
    \label{tab:measurements-ir-rotation}
\end{table}
